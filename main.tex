\documentclass[a4paper,12pt]{article}
\usepackage[dvipsnames]{xcolor}
\usepackage{emoji}
\usepackage{enumitem}
\usepackage{natbib}
\usepackage[backend=biber]{biblatex}
\addbibresource{references.bib}

\title{\textbf{Project Status Report} \emoji{rocket}}
\author{Arthur Wuhrmann}
\date{\today}
\usepackage{hyperref}

\begin{document}

\maketitle

\paragraph{Goal (to refine)} We discussed on 21/02, and it seems like many approaches could be taken and it is important to be specific about what I want to achieve. 
The goal is to perform exact or near-exact matching from the output of the LLM to its training data. We will first cover a word-specific topic to make the task simpler, like chemistry. Different prompt can be used to try to match some input, like using "According to academia, [...]" as a prefix. The matching can be done either on Google, or by indexing some parts of the training data (if known). 

This means we will not look for general model biases (like biased representations of jobs etc.), because it is less tractable. We might also limit ourselves to certain training data / models, for computational reasons.

\section*{Reading \emoji{books}}
\subsection*{Papers Read}
\begin{itemize}
    \item \textbf{Paper 1:} \citeauthor{bender_dangers_2021} (\citeyear{bender_dangers_2021}) - \textit{\citetitle{bender_dangers_2021}} \emoji{white-check-mark}
        \item \textbf{Paper 2:} \citeauthor{elazar_whats_2024} (\citeyear{elazar_whats_2024}) - \textit{\citetitle{elazar_whats_2024}} \emoji{white-check-mark}

    \item \textbf{Paper 3:} \citeauthor{marone_data_2023} (\citeyear{marone_data_2023}) - \textit{\citetitle{marone_data_2023}} \emoji{white-check-mark}

    \item \textbf{Paper 4:} \citeauthor{mirzadeh_gsm-symbolic_2024} (\citeyear{mirzadeh_gsm-symbolic_2024}) - \textit{\citetitle{mirzadeh_gsm-symbolic_2024}} \emoji{white-check-mark}
    \item \textbf{Paper 5:} \citeauthor{lee_deduplicating_2022} (\citeyear{lee_deduplicating_2022}) - \textit{\citetitle{lee_deduplicating_2022}}: Shows different approaches to removing duplicates \emoji{white-check-mark}

\end{itemize}


\subsection*{Main Takeaways}
\begin{itemize}
    \item Type of indexing to try contains Bloom filters\cite{marone_data_2023}, suffix arrays \cite{lee_deduplicating_2022}, grepping\cite{marone_data_2023}, Jaccard index for partial matching \cite{lee_deduplicating_2022}
    \item Unsafe models to do preliminary analysis : \href{https://huggingface.co/spaces/DontPlanToEnd/UGI-Leaderboard}{HuggingFace link}. However, harder to find a pair unsafe model - dataset, where the dataset is properly labelled and where some parts only can be downloaded (like the chemistry part)
    \item However, this approach would also mean that we probably won't have exact matching. Thus, we should be careful and 
\end{itemize}



\section*{Coding \emoji{computer}}
Available on \href{https://github.com/Reliable-Information-Lab-HEVS/HAIDI-Graphs}{GitHub}. 
\subsection*{To be done until Mar 01 / Mar 08}
\begin{itemize}
    \item [\color{ForestGreen}{DONE}] Got model inference on cluster (Put right parameters, as Andrei suggested)
    \item [\color{Goldenrod}{WIP}] Create diverse dangerous prompts.
    \item [\color{Goldenrod}WIP] Get the perplexity for each token of the answer and make nice visualization plots
    \item [\color{BrickRed}SOON] Isolate potential matches from the outputs
    \item [\color{BrickRed}SOON] Implement (near-)grep and bloom filters on the chemistry part of The Pile
    \item [\color{BrickRed}SOON] Perform matching and evaluate the performance
\end{itemize}

\section*{Writing \emoji{pencil}}
\textit{Not yet}

\section*{Meetings}
\begin{itemize}
    \item \textbf{2025/02/10}: First meeting. Talked about the project in general, mentioned what techniques existed. We agreed that we should try a smaller size project, and might have to target smaller datasets, only on specific areas for example.  
    \item \textbf{2025/02/21}: We reviewed some of the literature, and try to specify clearly which goal we would achieve (see the beginning of the report). We stated clear next steps, i.e. try a first implementation. Infer some dangerous prompt, try to find it back. For now, "simple" technique like (near-)grepping or bloom filters. Also, look at the perplexity of the model when answering.  
    \item \textbf{2025/02/27}: Meeting with Andrei, Alexander, Anastasiia, Ines and Leo. We cleared differentiated the points of our projects, and got some feedback. I need to log more what I find. We mentioned some of the difficulties I would encounter (non-exact matching, size of data, ...)
    
\end{itemize}

\printbibliography
\end{document}

